\documentclass[aspectratio=169]{beamer}

\usepackage{caption}

% preamble that loads packages, colors and macros and does general document set-up

% packages
\usepackage[utf8]{inputenc}
\usepackage[T1]{fontenc}
\usepackage[english]{babel} % i18n
% typesetting
% \usepackage{ebgaramond}
\usepackage{hyperref}
\usepackage{bookmark}
\usepackage{url}
\usepackage{setspace}  % double spacing
\usepackage{textcomp}  % euro sign
\usepackage[parfill]{parskip}
\usepackage{lscape}
\usepackage{xspace} % adds space after a macro, unless it is followed by certain punctuation marks.
\usepackage{marginfix}
% math typesetting
\usepackage{amsmath}
\usepackage{amssymb}
\usepackage{amsfonts}
\usepackage{mathtools}
\usepackage{bm}
\usepackage[binary-units]{siunitx}
% tables
\usepackage{tabulary}
\usepackage{booktabs}
\usepackage{multirow}
\usepackage{longtable} 
% figures
\usepackage{graphicx}
\usepackage{pgf}  % PGF support
\usepackage{svg}  % SVG support
\usepackage{tikz}
\usepackage{tikzscale}
\usepackage{pgfplots}
% algorithms
\usepackage{minted}  % code listing
\usepackage[linesnumbered,ruled,vlined]{algorithm2e}  % algorithms
% colors
\usepackage{color}
\usepackage{xcolor}
\usepackage{soul}
% front matter
\usepackage[nottoc]{tocbibind}
\usepackage[acronym,nomain,toc]{glossaries}
% bibliography
\usepackage{natbib}
% referencing
\usepackage[capitalise]{cleveref}

% colors
% \definecolor{urlcolor}{RGB}{0,37,148}
\definecolor{urlcolor}{RGB}{18,18,129}
\definecolor{linkcolor}{RGB}{123,25,25}
\definecolor{highlightcolor}{RGB}{123,25,25}
% \definecolor{citecolor}{RGB}{89,89,89}
\definecolor{citecolor}{RGB}{50,98,50}
\newcommand{\highlight}[1]{\textcolor{highlightcolor}{#1}}

\definecolor{red}{RGB}{211, 47, 47}
\definecolor{pink}{RGB}{194, 24, 91}
\definecolor{purple}{RGB}{123, 31, 162}
\definecolor{indigo}{RGB}{48, 63, 159}
\definecolor{blue}{RGB}{25, 118, 210}
\definecolor{cyan}{RGB}{0, 151, 167}
\definecolor{green}{RGB}{56, 142, 60}
\definecolor{lime}{RGB}{175, 180, 43}
\definecolor{yellow}{RGB}{251, 192, 45}
\definecolor{orange}{RGB}{245, 124, 0}

\sethlcolor{yellow}

% macros
\DeclareMathOperator*{\argmin}{arg\,min}
\DeclareMathOperator*{\argmax}{arg\,max}
\DeclareMathOperator*{\score}{score}

\newcommand*{\B}[1]{\ifmmode\bm{#1}\else\textbf{#1}\fi}
\newcommand*{\I}[1]{\ifmmode\mathit{#1}\else\textit{#1}\fi}

% vectors
\newcommand{\vect}[1]{\B{#1}}
\newcommand{\xvect}{\vect{x}\xspace}
\newcommand{\yvect}{\vect{y}\xspace}
\newcommand{\cvect}{\vect{c}\xspace}
\newcommand{\hvect}{\vect{h}\xspace}
\newcommand{\svect}{\vect{s}\xspace}
\newcommand{\Wvect}{\vect{W}\xspace}

% math operators
\newcommand{\logg}[1]{\log{\left({#1}\right)}}
\newcommand{\expp}[1]{\exp{\left({#1}\right)}}
\newcommand{\tanhh}[1]{\tanh{\left({#1}\right)}}
\newcommand{\alignn}[1]{\align\left({#1}\right)}
\newcommand{\scoree}[1]{\score\left({#1}\right)}
\newcommand{\loss}[1]{\mathcal{L}\left({#1}\right)}
\newcommand{\softmax}[1]{\sigma\left({#1}\right)}
\newcommand{\ce}[1]{\mathcal{H}\left({#1}\right)}

% functions with brackets
\newcommand{\floor}[1]{\left\lfloor {#1} \right\rfloor}
\newcommand{\ceil}[1]{\left\lceil{#1}\right\rceil}
\newcommand{\round}[1]{\left\lfloor{#1}\right\rceil}
\newcommand{\norm}[1]{\left\lVert{#1}\right\rVert}
\newcommand{\intervalcc}[1]{\left[{#1}\right]}
\newcommand{\intervalco}[1]{\left[{#1}\right)}
\newcommand{\intervaloc}[1]{\left({#1}\right]}
\newcommand{\intervaloo}[1]{\left({#1}\right)}
\newcommand{\repeating}[1]{\overline{#1}}
\newcommand{\sizeof}[1]{\left|{#1}\right|}

% other
\newcommand{\reals}[1]{\mathbb{R}^{#1}}
\newcommand{\dataset}{\mathcal{D}\xspace}
\newcommand{\corpus}{\dataset_\text{Corpus}\xspace}

\newcommand{\redtimes}{\color{red}$\times$\xspace}
\newcommand{\greencheck}{\color{green}$\checkmark$\xspace}

\newcommand{\carbon}{CO$_{\text{2}}$\xspace}

\newcommand{\bertbase}{BERT$_{\text{BASE}}$\xspace}
\newcommand{\bertlarge}{BERT$_{\text{LARGE}}$\xspace}
\newcommand{\bertstudent}{BERT$_{\text{STUDENT}}$\xspace}
\newcommand{\bertsmall}{BERT$_{\text{SMALL}}$\xspace}
\newcommand{\berttiny}{BERT$_{\text{TINY}}$\xspace}
\newcommand{\bertthree}{BERT$_3$\xspace}
\newcommand{\bertsix}{BERT$_6$\xspace}
\newcommand{\berttwelve}{BERT$_12$\xspace}

\newcommand{\cls}{\texttt{[CLS]}\xspace}
\newcommand{\sep}{\texttt{[SEP]}\xspace}
\newcommand{\mask}{\texttt{[MASK]}\xspace}

\hypersetup{
    colorlinks,
    bookmarksdepth=section,
    linkcolor=linkcolor,
    citecolor=citecolor,
    urlcolor=urlcolor
}

\titleformat{\chapter}
  {\color{highlightcolor}\Huge\rmfamily\itshape}    % label
  {\color{highlightcolor}\itshape\HUGE\thechapter~} % number
  {0pt}                                             % horizontal separation between label and title body
  {}                                                % before the title body
  []                                                % after the title body

\titlespacing*{\section}{0pt}{*10}{2.3ex plus .2ex}
\titlespacing*{\subsection}{0pt}{*5}{1.5ex plus.2ex}
  
\setcounter{tocdepth}{1}  % ToC depth (sections)
\setcounter{secnumdepth}{2}  % section numbering depth, includes subsections

% page numbering
\fancypagestyle{plain}{
  \fancyhf{}
  \fancyhead[RE,RO]{\smallcaps{\newlinetospace{\plaintitle}}\quad\thepage}
}

\makeatletter
% Paragraph indentation and separation for normal text
\renewcommand{\@tufte@reset@par}{%
  \setlength{\RaggedRightParindent}{0.0pc}%
  \setlength{\JustifyingParindent}{0pt}% % parindent
  \setlength{\parindent}{0pt}%
  \setlength{\parskip}{5pt}%
}
\@tufte@reset@par

% Paragraph indentation and separation for marginal text
\renewcommand{\@tufte@margin@par}{%
  \setlength{\RaggedRightParindent}{0.0pc}%
  \setlength{\JustifyingParindent}{0.0pc}%
  \setlength{\parindent}{24pt}%
  \setlength{\parskip}{10pt}%
}
\makeatother

\glsdisablehyper
\makeglossaries
\newacronymstyle{long-short-emph}%
{%
  \ifglshaslong{\glslabel}%
  {% acronym
   % This is an adaptation of \glsgenacfmt
    \ifdefempty\glscustomtext
    {%
      \ifglsused\glslabel
      {%
        \glsifplural
        {%
          \glscapscase
          {%
            \acronymfont{\glsentryshortpl{\glslabel}}\glsinsert
          }%
          {%
            \acronymfont{\Glsentryshortpl{\glslabel}}\glsinsert
          }%
          {%
            \mfirstucMakeUppercase
              {\acronymfont{\glsentryshortpl{\glslabel}}\glsinsert}%
          }%
        }%
        {%
          \glscapscase
          {%
            \acronymfont{\glsentryshort{\glslabel}}\glsinsert
          }%
          {%
            \acronymfont{\Glsentryshort{\glslabel}}\glsinsert
          }%
          {%
            \mfirstucMakeUppercase
              {\acronymfont{\glsentryshort{\glslabel}}\glsinsert}%
          }%
        }%
      }%
      {%
        \glsifplural
        {%
          \glscapscase
          {%
            \emph{\glsentrylongpl{\glslabel}\glsinsert}\space
            (\firstacronymfont{\glsentryshortpl{\glslabel}})%
          }%
          {%
            \emph{\Glsentrylongpl{\glslabel}{\glsinsert}}\space
            (\firstacronymfont{\glsentryshortpl{\glslabel}})%
          }%
          {%
            \mfirstucMakeUppercase
              {\emph{\glsentrylongpl{\glslabel}{\glsinsert}}\space
               (\firstacronymfont{\glsentryshortpl{\glslabel}})}%
          }%
        }%
        {%
          \glscapscase
          {%
            \emph{\glsentrylong{\glslabel}\glsinsert}\space
            (\firstacronymfont{\glsentryshort{\glslabel}})%
          }%
          {%
            \emph{\Glsentrylong{\glslabel}\glsinsert}\space
            (\firstacronymfont{\glsentryshort{\glslabel}})%
          }%
          {%
            \mfirstucMakeUppercase
             {\emph{\glsentrylong{\glslabel}\glsinsert}\space
             (\firstacronymfont{\glsentryshort{\glslabel}})}%
          }%
        }%
      }%
    }%
    {%
      \glscustomtext
    }%
  }%
  {% regular term
    \ifglsused{\glslabel}{\glsgenentryfmt}{\emph{\glsgenentryfmt}}%
  }%
}%
{%
  \GlsUseAcrStyleDefs{long-short}%
}
\setacronymstyle{long-short-emph}
\loadglsentries[main]{glossary}

\usetikzlibrary{
    3d,
    arrows,
    arrows.meta,
    backgrounds,
    bending,
    calc,
    chains,
    decorations.pathmorphing,
    decorations.pathreplacing,
    graphs,
    matrix,
    patterns,
    positioning,
    quotes,
    shapes.arrows,
    shapes.geometric,
    shapes.misc,
}
\tikzset{>=latex}

\usepgfplotslibrary{
    groupplots,
    dateplot,
}
\pgfplotsset{compat=newest}

\DeclareSIUnit{\nothing}{\relax}
\DeclareSIUnit{\billion}{\text{B}}

\hyphenation{Wiki-pedia}

\newcommand{\noopsort}[2]{#2}

% UvA colors
\definecolor{UvA_Black}{RGB}{31, 29, 33}
\definecolor{UvA_Red}{RGB}{188, 0, 49}
\definecolor{UvA_White}{RGB}{255, 255, 255}
\definecolor{UvA_Grey15}{RGB}{220, 221, 222}
\definecolor{UvA_Grey8}{RGB}{235, 235, 236}
\definecolor{UvA_FNWI_Purple}{RGB}{117, 27, 104}

\setbeamercolor{emph}{fg=UvA_FNWI_Purple}
\renewcommand<>{\emph}[1]{%
  {\usebeamercolor[fg]{emph}\only#2{\itshape}#1}%
}

\DeclareRobustCommand{\VAN}[3]{#2} % set up for citation

\hypersetup{
    colorlinks,
    linkcolor=UvA_FNWI_Purple,
    citecolor=UvA_Black,
    urlcolor=UvA_FNWI_Purple
}

% Metropolis beamer theme
\usetheme[numbering=fraction, block=fill]{metropolis}

% Metropolis beamer theme colors
\setbeamercolor{progress bar}{fg=UvA_FNWI_Purple}
\setbeamercolor{normal text}{fg=UvA_Black, bg=UvA_White}
\setbeamercolor{palette primary}{fg=UvA_White, bg=UvA_FNWI_Purple}

% font
\usefonttheme{serif}
\setbeamerfont{table}{size=\scriptsize}
\setbeamerfont{caption}{size=\tiny}


% metadata
\title{Distilling BERT: Transfer Dataset Size and Composition Effects}
\date{\today}
\author{Steven van de Graaf}
\institute{\textsc{University of Amsterdam}}

\begin{document}
    \maketitle
    \section{Introduction}
    \begin{frame}{Motivation}
        \begin{itemize}
            \item ``ImageNet moment'' for \gls{nlp} thanks to Transformers
            \item Transformers are getting ever bigger:
        \end{itemize}
        \begin{table}[ht!]
            \scriptsize
            \centering
            \begin{tabular}{l | l l r }
                \toprule
                \B{Paper} & \B{Date} & \B{Name} & \B{Number of Parameters} \\
                \midrule
                \citet{radford2018improving} & 06-2018 & GPT & \num{110000000} \\
                \citet{devlin2018bert} & 10-2018 & \bertlarge & \num{320000000} \\
                \citet{radford2019language} & 02-2019 & GPT-2 & \num{1542000000} \\
                \citet{shoeybi2019megatron} & 09-2019 & Megatron & \num{8300000000} \\
                \citet{raffel2019exploring} & 10-2019 & T5-11B & \num{11000000000} \\
                \citet{microsoft2020turing} & 02-2020 & Turing-NLG & \num{17000000000} \\
                \citet{brown2020language} & 05-2020 & GPT-3 & \num{175000000000} \\
                \citet{fedus2021switch} & 01-2021 & Switch-C & \num{1571000000000} \\
                \bottomrule
            \end{tabular}
            \caption[Sizes of recent Transformer(-based) architectures]{Sizes of recent Transformer(-based) architectures in terms of the number of parameters.}
        \end{table}
    \end{frame}
    
    \begin{frame}{Motivation (cont.)}
        \begin{itemize}
            \item Increasing size comes at significant costs in terms of \emph{sustainability} and \emph{accessibility}.
            \item Increasing importance of methods that make \glspl{nn} more efficient (``Green AI'').
            \item \gls{kd} is one such promising method to ``compress'' \glspl{nn}.
        \end{itemize}
    \end{frame}
    
    \begin{frame}{Knowledge Gap}
        \begin{itemize}
            \item In the last two years or so, there have been various efforts to apply \gls{kd} to Transformer-based architectures.
            \item The resulting ``student networks'' are both \emph{smaller} and \emph{faster}, while still achieving \emph{comparable performance}.
            \item While promising, this new line of research is still immature, as reflected by the lack of established best-practices when it comes to the various aspects of \gls{kd}.
        \end{itemize}
    \end{frame}
    
    \begin{frame}{Scope}
        \begin{itemize}
            \item While the biggest \glspl{nn} (containing hundreds of billions of parameters) stand to gain the most from being compressed, this research focuses its efforts solely on the popular \emph{\bertbase} network by \citet{devlin2018bert}.
            \item When it comes to what aspects of the \gls{kd} process to investigate, this research has settled on the \emph{transfer dataset} $\dataset_T$ used during pre-training.
        \end{itemize}
    \end{frame}
    
    \begin{frame}{Research Questions}
        \noindent\begin{tabular}{p{0.075\linewidth} p{0.875\linewidth}}
        \B{RQ$_1$} & What are the effects of the \emph{size} of the transfer dataset $\dataset_T$ used during pre-training in the Knowledge Distillation process as applied to \bertbase, measured in performance on downstream performance task(s)? \\
        \addlinespace
        \B{RQ$_2$} & What are the effects of the \emph{composition} of the transfer dataset $\dataset_T$ used during pre-training in the Knowledge Distillation as applied to \bertbase, measured in performance on downstream performance task(s)?
    \end{tabular}
    \end{frame}
    
    
    
    \section{Background}
    \begin{frame}{\bertbase}
        \begin{itemize}
            \item Almost three years ago, \citet{devlin2018bert} introduced their new language representation \gls{nn} \gls{bert}.
            \item \gls{bert} is a Transformer-based architecture that uses (only) the \emph{Encoder} stack of the Transformer architecture \citep{vaswani2017attention}.
            \item \gls{bert} was pre-trained using a \gls{mlm} objective:\\
            It doesn't take a \mask to figure this out.
        \end{itemize}
    \end{frame}
    
    \begin{frame}{Knowledge Distillation}
        \begin{itemize}
            \item First introduced by \citet{bucil2006model} and popularized by \citet{hinton2015distilling}.
            \item A smaller (untrained) \emph{student network} is trained to mimic a larger, pre-trained \emph{teacher network} by means of \emph{knowledge transfer}.
        \end{itemize}
    \end{frame}
    
    \begin{frame}{Soft target loss}
        The student network is trained on some \emph{transfer dataset} $\dataset_T$ to minimize a loss function in which the target is the distribution of class probabilities as output by the teacher network (\emph{soft target loss}):
        \begin{align}
            \label{eq:soft_target_loss}
            \mathcal{L}_{\text{soft}} = \loss{\vect{z}_s, \vect{z}_t} &= KL\left( \softmax{\vect{z}_s} || \softmax{\vect{z}_t} \right).
        \end{align}
        Advantages:
        \begin{itemize}
            \item The data in $\dataset_T$ can be entirely unlabeled.
            \item Provides a richer signal during training when compared to only using the ground truth label as your target.
        \end{itemize}
    \end{frame}
    
    \begin{frame}{Hard target loss}
        When some or all of the data in $\dataset_T$ is labeled, \citet{hinton2015distilling} have found it beneficial to train the student network to also predict the ground truth labels $\vect{y}$ (\emph{hard target loss}):
        \begin{align}
            \label{eq:hard_target_loss}
            \mathcal{L}_{\text{hard}} = \loss{\vect{z}_s, \vect{y}} &= \ce{\softmax{\vect{z}_s}, \vect{y}}.
        \end{align}
    \end{frame}
    
    \begin{frame}{Combined loss function}
        Combining the loss functions defined in \cref{eq:soft_target_loss,eq:hard_target_loss} yields the following combined loss function:
        \begin{align}
            \label{eq:combined_kd_loss}
            \mathcal{L}_{\text{combined}} &= \loss{\vect{z}_s, \vect{z}_t, \vect{y}}, \nonumber \\
            &= \alpha \cdot KL\left( \softmax{\vect{z}_s} || \softmax{\vect{z}_t} \right) \nonumber \\
            &\quad + \beta \cdot \ce{\softmax{\vect{z}_s}, \vect{y}}.
        \end{align}
    \end{frame}
    
    
    
    \section{Related Work}
    \begin{frame}{Variables}
        \gls{kd} as described by \citet{hinton2015distilling} provides only a general recipe that leaves much to be specified, such as:
        \begin{itemize}
            \item What \gls{nn} architecture(s) do you use?
            \item What transfer dataset do you use?
            \item What composition of the combined loss function do you use?
        \end{itemize}
    \end{frame}
        
    \begin{frame}{Variables (cont.)}
       For Transformer(-based) architectures specifically, at least one more important variable can be added to the mix:
        \begin{itemize}
            \item At what stage (pre-training, fine-tuning or both) is \gls{kd} applied?
        \end{itemize}
        
        
        The last variable to consider is more generally applicable to all scientific research:
        \begin{itemize}
            \item How do you evaluate your approach / method?
        \end{itemize}
    \end{frame}
    
    \begin{frame}{Summary (condensed)}
        \begin{table}[ht!]
            \tiny
            \centering
            \begin{tabular}{l | l l c l }
                \toprule
        \B{Paper} & \B{Teacher network} & \B{Stage} & \B{$\dataset_T$} & \B{Evaluation} \\
        \midrule
        \citet{tang2019distilling} & \bertlarge & FT & \redtimes & Subset of GLUE  \\
        \citet{liu2019improving} & MT-D\gls{nn} & FT & \redtimes & GLUE \\
        \citet{yang2019model} & \bertbase & FT & \redtimes & DeepQA \\
        \citet{sun2019patient} & \bertbase & FT & \redtimes & Subset of GLUE \\
        \citet{jiao2019tinybert} & \bertbase & Both & ``large-scale text corpus'' & GLUE \\
        \citet{turc2019well} & \bertbase & FT & TBC + EnWiki & Subset of GLUE \\
        \citet{sanh2019distilbert} & \bertbase & PT & TBC + EnWiki & GLUE \& SQuAD \\
        \citet{sun2020mobilebert} & IB-\bertlarge & PT & TBC + EnWiki & GLUE \& SQuAD \\
        \bottomrule
            \end{tabular}
            \caption{Condensed summary of related works along the questions posed previously.}
        \end{table} 
    \end{frame}
    
    
    
    \section{Experimental Design}
    \begin{frame}{\gls{nn} Architecture}
        Following the examples of \citet{sun2019patient,sanh2019distilbert}, we use \emph{\bertbase} for our \emph{teacher network} and a shallower version of it, using 6 layers instead of 12, for our \emph{student network}, which we refer to as \emph{\bertstudent}.

        \begin{table}[h]
            \scriptsize
            \centering
            \begin{tabular}{l|c|c|c}
                \toprule
                \B{Network} & \B{$\#P$} & \B{Size-on-disk} & \B{Inference time} \\
                \midrule
                \bertbase & \SI{109.5}{\mega\nothing} ($1.0 \times$) & \SI{417}{\mega\byte} ($1.0 \times$) & \SI{1.47}{\second} ($1.0 \times$) \\
                \bertstudent & \SI{66.4}{\mega\nothing} ($1.65 \times$) & \SI{253}{\mega\byte} ($1.65 \times$) & \SI{0.80}{\second} ($1.83 \times$) \\
                \bottomrule
            \end{tabular}
            \caption{Comparison of network size (in $\#P$ and size-on-disk) and inference time (in s) between our teacher network, \bertbase, and our student network, \bertstudent.}
            \label{tab:network_comparison}
        \end{table}
    \end{frame}
    
    \begin{frame}{Pre-training}
        \begin{itemize}
            \item Like \citet{jiao2019tinybert,sanh2019distilbert,sun2020mobilebert}, we apply \gls{kd} during pre-training.
            \item For our composite loss function, we combine the hard target loss and soft target loss seen previously, with a cosine embedding loss that leverages the knowledge encoded in intermediate layers:
        \end{itemize}
        
        \begin{align}
            \label{eq:our_combined_loss}
            \mathcal{L} &= \alpha_\text{soft} \cdot \mathcal{L}_{\text{soft}} + \alpha_\text{hard} \cdot \mathcal{L}_{\text{hard}} + \alpha_\text{cos} \cdot \mathcal{L}_{\text{cos}}, \\
            &= \alpha_\text{soft} KL\left( \softmax{\vect{z}_s} || \softmax{\vect{z}_t} \right) \nonumber \\
            &\quad+ \alpha_\text{hard} \cdot \ce{\softmax{\vect{z}_s}, \vect{y}} \nonumber \\
            &\quad+ \alpha_\text{cos} \cdot \left( 1 - \cos{\left( \hvect_s, \hvect_t \right)} \right). \nonumber
        \end{align}
    \end{frame}
    
    \begin{frame}{Data}
        Following the pre-training procedures of both \bertbase \citep{devlin2018bert} and DistilBERT \citep{sanh2019distilbert}, we use a concatenation of (a replica of) the \gls{tbc} dataset \citep{zhu2015aligning} and \emph{English Wikipedia} for our composite pre-training corpus.
    \end{frame}
    
    \begin{frame}{Toronto BookCorpus dataset}
        \begin{itemize}
            \item The \gls{tbc} dataset \citep{zhu2015aligning} is no longer publicly available.
            \item \href{https://www.smashwords.com/}{Smashwords} still exists, however.
            \item Following the procedure of \citet{graaf2019replicating}, we have developed and released \href{https://github.com/sgraaf/Replicate-Toronto-BookCorpus}{\emph{Replicate Toronto BookCorpus}}, a collection of scripts that achieve the following 3 steps:
            \begin{enumerate}
            \item Finding those books that meet our criteria.
            \item Downloading these books.
            \item Pre-processing these books.
        \end{enumerate}
        \end{itemize}
    \end{frame}
    
    \begin{frame}{Toronto BookCorpus dataset (cont.)}
        Following these steps, we were able to create a faithful replica of the \gls{tbc} dataset:

        \begin{table}[ht]
            \scriptsize
            \centering
            \begin{tabular}{l|c|c|c|c|c}
                \toprule
                \B{Dataset} & \B{Size} & \B{N\textsuperscript{\underline{o}} sentences} & \B{N\textsuperscript{\underline{o}} words} & \B{mean w.p.s.} & \B{median w.p.s.} \\
                \midrule
                \gls{tbc} \citep{zhu2015aligning} & \SI{4.7}{\giga\byte} & \SI{74.0}{\mega\nothing} & \SI{984.8}{\mega\nothing} & $13.3$ & $11$ \\
                \midrule
                \gls{tbc} replica & \SI{5.1}{\giga\byte} & \SI{65.4}{\mega\nothing} & \SI{897.1}{\mega\nothing} & $13.7$ & $11$ \\
                \bottomrule
            \end{tabular}
            \caption{Comparison of datasets in terms of size and other important statistics. ``w.p.s.'' stands for ``words per sentence''.}
        \end{table}
    \end{frame}
    
    \begin{frame}{English Wikipedia}
        \begin{itemize}
            \item While used at least equally frequently in \gls{lm} pre-training, English Wikipedia also requires much pre-processing.
            \item For this, we follow the procedure of \citet{graaf2019preprocessing}:
            \begin{enumerate}
                \item Download the latest ``dump'' of all pages and articles of English Wikipedia.
                \item Extract only the text passages (ignoring any lists, tables and headers).
                \item Tokenize the sentences.
                \item Concatenate all pages and articles into a single text-file.
            \end{enumerate}
        \end{itemize}
    \end{frame}
    
    \begin{frame}{$\corpus$}
        \begin{itemize}
            \item What we refer to as ``our corpus'' or $\corpus$ is now a concatenation of our \gls{tbc} replica and English Wikipedia.
            \item $\corpus$ consists of \SI{109.3}{\mega\nothing} \emph{sequences} or \SI{4.0}{\billion} \emph{tokens} in total.
        \end{itemize}
    \end{frame}
    
    \begin{frame}{Baselines}
        In order to verify the significance of our results, we compare the performance of \bertstudent to two baselines: 
        \begin{enumerate}
            \item \B{\bertbase} \citep{devlin2018bert}
            \item \B{DistilBERT} \citep{sanh2019distilbert}
        \end{enumerate}
    \end{frame}
    
    
    
    \section{Results}
    \subsection{Size of Transfer Dataset $\dataset_T$}
    \begin{frame}{Methods}
        \begin{itemize}
            \item We take three random samples (without replacement) from our corpus $\corpus$ using three different sample sizes in order to obtain three differently sized transfer datasets $\dataset_T^N$.
            \item The sample sizes used for the random sampling from our corpus are proportional to its size (\SI{109.3}{\mega\nothing} sequences or \SI{4.0}{\billion} tokens) by different orders of magnitude: 
            \begin{enumerate}
                \item We take a first random sample that is $10$\% of its size (\SI{10.9}{\mega\nothing} sequences or \SI{402.6}{\mega\nothing} tokens)
                \item We take a second random sample that is $1$\% of its size (\SI{1.1}{\mega\nothing} sequences or \SI{40.3}{\mega\nothing} tokens)
                \item We take a third and final random sample that is $0.1$\% of its size (\SI{109.4}{\kilo\nothing} sequences or \SI{4.0}{\mega\nothing} tokens)
            \end{enumerate}
        \end{itemize}
    \end{frame}
    
    \begin{frame}{Methods (cont.)}
        \begin{table}[ht!]
            \scriptsize
            \centering
            \begin{tabular}{l|c|c|c}
                \toprule
                \B{Dataset} & \B{Size} & \B{N\textsuperscript{\underline{o}} sequences} & \B{N\textsuperscript{\underline{o}} tokens} \\
                \midrule
                $\corpus$ & \SI{20.0}{\giga\byte} & \SI{109.3}{\mega\nothing} & \SI{4.0}{\billion} \\
                \midrule
                $\dataset_T^{10\%}$ & \SI{2.0}{\giga\byte} & \SI{10.9}{\mega\nothing} & \SI{402.6}{\mega\nothing} \\
                \midrule
                $\dataset_T^{1\%}$ & \SI{204.8}{\mega\byte} & \SI{1.1}{\mega\nothing} & \SI{40.3}{\mega\nothing} \\
                \midrule
                $\dataset_T^{0.1\%}$ & \SI{20.6}{\mega\byte} & \SI{109.4}{\kilo\nothing} & \SI{4.0}{\mega\nothing} \\
                \bottomrule
            \end{tabular}
            \caption{Comparison between our corpus and our transfer datasets (which are differently sized random samples of our corpus).}
        \end{table}
    \end{frame}
    
    \begin{frame}{Results - GLUE}
        \begin{table}[ht!]
            \scriptsize
            \centering
            \begin{tabular}{l c | c }
                \toprule
                \B{Network} & \B{Transfer dataset} & \B{Score}  \\
                \midrule
                \multicolumn{3}{l}{\emph{Ours:}} \\
                \multirow{4}{*}{\bertstudent} & $\dataset_T^{10\%}$ & \B{72.5} \\
                & $\dataset_T^{1\%}$ & \B{72.4} \\
                & $\dataset_T^{0.1\%}$ & \B{70.0} \\
                % & $\dataset_T^0$ & \B{60.2} & 0.0 & 79.1 & 75.7 & 85.3 & 86.7 & 54.2 & 89.3 & 15.1 & 56.3 \\
                \midrule
                \multicolumn{3}{l}{\emph{Baselines:}} \\
                \multicolumn{2}{l|}{\bertbase \citep{devlin2018bert}} & \B{74.9} \\
                \multicolumn{2}{l|}{DistilBERT \citep{sanh2019distilbert}} & \B{74.3} \\
                \bottomrule 
            \end{tabular}
            \caption{Comparison of performance on the GLUE benchmark using transfer datasets of various sizes. Results of \bertbase and DistilBERT as reported by \citet{sanh2019distilbert}.}
        \end{table}
    \end{frame}
    
    \begin{frame}{Results - SQuAD}
        \begin{table}[ht!]
            \scriptsize
            \centering
            \begin{tabular}{l c | c c }
                \toprule
                \multirow{2}{*}{\B{Network}} & \multirow{2}{*}{\B{Transfer dataset}} & \multicolumn{2}{c}{\B{Score}} \\
                & & EM & \B{F1} \\
                \midrule
                \multicolumn{4}{l}{\emph{Ours:}} \\
                \multirow{4}{*}{\bertstudent} & $\dataset_T^{10\%}$ & 56.9 & \B{68.5} \\
                & $\dataset_T^{1\%}$ & 54.7 & \B{66.4} \\
                & $\dataset_T^{0.1\%}$ & 45.0 & \B{56.5} \\
                % & $\dataset_T^0$ & 42.6 & 54.9 \\
                \midrule
                \multicolumn{4}{l}{\emph{Baselines:}} \\
                \multicolumn{2}{l|}{\bertbase \citep{devlin2018bert}} & 73.9 & \B{82.3} \\
                \multicolumn{2}{l|}{DistilBERT \citep{sanh2019distilbert}} & 69.6 & \B{78.7} \\
                \bottomrule
            \end{tabular}
            \caption{Comparison of performance on SQuAD using transfer datasets of various sizes. Results of \bertbase and DistilBERT as reported by \citet{sanh2019distilbert}.}
            \label{tab:amount_of_data_squad_results}
        \end{table}
    \end{frame}
    
    \begin{frame}{Results}
        \begin{figure}[ht!]
            \begin{center}
                \input{figures/07-results/amount_of_data_barplot.pgf}
            \end{center}
            \caption{Plot of performance of \bertstudent on GLUE and SQuAD using transfer datasets of various sizes.}
            \label{fig:amount_of_data_barplot}
        \end{figure}
    \end{frame}
    
    
    \subsection{Composition of Transfer Dataset $\dataset_T$}
    \begin{frame}{Methods}
        We create two new transfer datasets $\dataset_T$ of different compositions by adding ``noise'' to $\dataset_T^{10\%}$:
        \begin{enumerate}
            \item The first new transfer dataset we create is an randomization of $\dataset_T^{10\%}$, wherein each sequence of tokens in $\dataset_T^{10\%}$ is randomized (i.e. shuffled). This dataset is labeled as $\dataset_T^{\text{Rand.}}$.
            \item The second new transfer dataset we create consists of entirely generated data (labeled as $\dataset_T^{\text{Gen.}}$). For this, we first compute two statistics of $\dataset_T^{10\%}$, which are then used to generate completely new sequences.
        \end{enumerate}
    \end{frame}
    
    \begin{frame}{Methods (cont.)}
        \begin{table}[ht!]
            \scriptsize
            \centering
            \begin{tabular}{c|c|c|c}
                \toprule
                \B{Transfer dataset} & \B{Size} & \B{N\textsuperscript{\underline{o}} sequences} & \B{N\textsuperscript{\underline{o}} tokens} \\
                \midrule
                $\dataset_T^{10\%}$ & \SI{2.0}{\giga\byte} & \SI{10.9}{\mega\nothing} & \SI{402.6}{\mega\nothing} \\
                \midrule
                $\dataset_T^{\text{Rand.}}$ & \SI{2.0}{\giga\byte} & \SI{10.9}{\mega\nothing} & \SI{402.6}{\mega\nothing} \\
                \midrule
                $\dataset_T^{\text{Gen.}}$ & \SI{2.0}{\giga\byte} & \SI{10.9}{\mega\nothing} & \SI{402.6}{\mega\nothing} \\
                \bottomrule
            \end{tabular}
            \caption{Comparison of the transfer datasets used in the current experiment (which are of different compositions).}
        \end{table}
    \end{frame}
    
    \begin{frame}{Results - GLUE}
        \begin{table}[ht!]
            \scriptsize
            \centering
            \begin{tabular}{c | c }
                \toprule
                \B{Transfer dataset} & \B{Score}  \\
                \midrule
                $\dataset_T^{10\%}$ & \B{72.5} \\
                $\dataset_T^{\text{Rand.}}$ & \B{64.8} \\
                $\dataset_T^{\text{Gen.}}$ & \B{59.2} \\
                \bottomrule 
            \end{tabular}
            \caption{Comparison of performance on the GLUE benchmark using transfer datasets of various compositions.}
        \end{table}
    \end{frame}
    
    \begin{frame}{Results - SQuAD}
        \begin{table}[ht!]
            \scriptsize
            \centering
            \begin{tabular}{c | c c }
                \toprule
                \multirow{2}{*}{\B{Transfer dataset}} & \multicolumn{2}{c}{\B{Score}} \\
                & EM & \B{F1} \\
                \midrule
                $\dataset_T^{10\%}$ & 56.9 & \B{68.5} \\
                $\dataset_T^{\text{Rand.}}$ & 8.7 & \B{16.2} \\
                $\dataset_T^{\text{Gen.}}$ &  27.0 & \B{37.3} \\
                \bottomrule
            \end{tabular}
            \caption{Comparison of performance on SQuAD using transfer datasets of various compositions.}
        \end{table}
    \end{frame}
    
    \begin{frame}{Results}
        \begin{figure}
            \begin{center}
                %% Creator: Matplotlib, PGF backend
%%
%% To include the figure in your LaTeX document, write
%%   \input{<filename>.pgf}
%%
%% Make sure the required packages are loaded in your preamble
%%   \usepackage{pgf}
%%
%% and, on pdftex
%%   \usepackage[utf8]{inputenc}\DeclareUnicodeCharacter{2212}{-}
%%
%% or, on luatex and xetex
%%   \usepackage{unicode-math}
%%
%% Figures using additional raster images can only be included by \input if
%% they are in the same directory as the main LaTeX file. For loading figures
%% from other directories you can use the `import` package
%%   \usepackage{import}
%%
%% and then include the figures with
%%   \import{<path to file>}{<filename>.pgf}
%%
%% Matplotlib used the following preamble
%%   \usepackage[utf8x]{inputenc}\usepackage[T1]{fontenc}\usepackage{amsmath}
%%   \usepackage{fontspec}
%%
\begingroup%
\makeatletter%
\begin{pgfpicture}%
\pgfpathrectangle{\pgfpointorigin}{\pgfqpoint{4.213420in}{2.604040in}}%
\pgfusepath{use as bounding box, clip}%
\begin{pgfscope}%
\pgfsetbuttcap%
\pgfsetmiterjoin%
\pgfsetlinewidth{0.000000pt}%
\definecolor{currentstroke}{rgb}{1.000000,1.000000,1.000000}%
\pgfsetstrokecolor{currentstroke}%
\pgfsetstrokeopacity{0.000000}%
\pgfsetdash{}{0pt}%
\pgfpathmoveto{\pgfqpoint{0.000000in}{0.000000in}}%
\pgfpathlineto{\pgfqpoint{4.213420in}{0.000000in}}%
\pgfpathlineto{\pgfqpoint{4.213420in}{2.604040in}}%
\pgfpathlineto{\pgfqpoint{0.000000in}{2.604040in}}%
\pgfpathclose%
\pgfusepath{}%
\end{pgfscope}%
\begin{pgfscope}%
\pgfsetbuttcap%
\pgfsetmiterjoin%
\definecolor{currentfill}{rgb}{1.000000,1.000000,1.000000}%
\pgfsetfillcolor{currentfill}%
\pgfsetlinewidth{0.000000pt}%
\definecolor{currentstroke}{rgb}{0.000000,0.000000,0.000000}%
\pgfsetstrokecolor{currentstroke}%
\pgfsetstrokeopacity{0.000000}%
\pgfsetdash{}{0pt}%
\pgfpathmoveto{\pgfqpoint{0.484445in}{0.443128in}}%
\pgfpathlineto{\pgfqpoint{4.213420in}{0.443128in}}%
\pgfpathlineto{\pgfqpoint{4.213420in}{2.555846in}}%
\pgfpathlineto{\pgfqpoint{0.484445in}{2.555846in}}%
\pgfpathclose%
\pgfusepath{fill}%
\end{pgfscope}%
\begin{pgfscope}%
\pgfpathrectangle{\pgfqpoint{0.484445in}{0.443128in}}{\pgfqpoint{3.728975in}{2.112717in}}%
\pgfusepath{clip}%
\pgfsetbuttcap%
\pgfsetmiterjoin%
\definecolor{currentfill}{rgb}{0.434314,0.146078,0.146078}%
\pgfsetfillcolor{currentfill}%
\pgfsetlinewidth{0.000000pt}%
\definecolor{currentstroke}{rgb}{0.000000,0.000000,0.000000}%
\pgfsetstrokecolor{currentstroke}%
\pgfsetstrokeopacity{0.000000}%
\pgfsetdash{}{0pt}%
\pgfpathmoveto{\pgfqpoint{0.608744in}{0.443128in}}%
\pgfpathlineto{\pgfqpoint{1.105941in}{0.443128in}}%
\pgfpathlineto{\pgfqpoint{1.105941in}{1.890741in}}%
\pgfpathlineto{\pgfqpoint{0.608744in}{1.890741in}}%
\pgfpathclose%
\pgfusepath{fill}%
\end{pgfscope}%
\begin{pgfscope}%
\pgfpathrectangle{\pgfqpoint{0.484445in}{0.443128in}}{\pgfqpoint{3.728975in}{2.112717in}}%
\pgfusepath{clip}%
\pgfsetbuttcap%
\pgfsetmiterjoin%
\definecolor{currentfill}{rgb}{0.434314,0.146078,0.146078}%
\pgfsetfillcolor{currentfill}%
\pgfsetlinewidth{0.000000pt}%
\definecolor{currentstroke}{rgb}{0.000000,0.000000,0.000000}%
\pgfsetstrokecolor{currentstroke}%
\pgfsetstrokeopacity{0.000000}%
\pgfsetdash{}{0pt}%
\pgfpathmoveto{\pgfqpoint{1.851736in}{0.443128in}}%
\pgfpathlineto{\pgfqpoint{2.348932in}{0.443128in}}%
\pgfpathlineto{\pgfqpoint{2.348932in}{0.785350in}}%
\pgfpathlineto{\pgfqpoint{1.851736in}{0.785350in}}%
\pgfpathclose%
\pgfusepath{fill}%
\end{pgfscope}%
\begin{pgfscope}%
\pgfpathrectangle{\pgfqpoint{0.484445in}{0.443128in}}{\pgfqpoint{3.728975in}{2.112717in}}%
\pgfusepath{clip}%
\pgfsetbuttcap%
\pgfsetmiterjoin%
\definecolor{currentfill}{rgb}{0.434314,0.146078,0.146078}%
\pgfsetfillcolor{currentfill}%
\pgfsetlinewidth{0.000000pt}%
\definecolor{currentstroke}{rgb}{0.000000,0.000000,0.000000}%
\pgfsetstrokecolor{currentstroke}%
\pgfsetstrokeopacity{0.000000}%
\pgfsetdash{}{0pt}%
\pgfpathmoveto{\pgfqpoint{3.094727in}{0.443128in}}%
\pgfpathlineto{\pgfqpoint{3.591924in}{0.443128in}}%
\pgfpathlineto{\pgfqpoint{3.591924in}{1.230923in}}%
\pgfpathlineto{\pgfqpoint{3.094727in}{1.230923in}}%
\pgfpathclose%
\pgfusepath{fill}%
\end{pgfscope}%
\begin{pgfscope}%
\pgfpathrectangle{\pgfqpoint{0.484445in}{0.443128in}}{\pgfqpoint{3.728975in}{2.112717in}}%
\pgfusepath{clip}%
\pgfsetbuttcap%
\pgfsetmiterjoin%
\definecolor{currentfill}{rgb}{0.146078,0.146078,0.434314}%
\pgfsetfillcolor{currentfill}%
\pgfsetlinewidth{0.000000pt}%
\definecolor{currentstroke}{rgb}{0.000000,0.000000,0.000000}%
\pgfsetstrokecolor{currentstroke}%
\pgfsetstrokeopacity{0.000000}%
\pgfsetdash{}{0pt}%
\pgfpathmoveto{\pgfqpoint{1.105941in}{0.443128in}}%
\pgfpathlineto{\pgfqpoint{1.603137in}{0.443128in}}%
\pgfpathlineto{\pgfqpoint{1.603137in}{1.976082in}}%
\pgfpathlineto{\pgfqpoint{1.105941in}{1.976082in}}%
\pgfpathclose%
\pgfusepath{fill}%
\end{pgfscope}%
\begin{pgfscope}%
\pgfpathrectangle{\pgfqpoint{0.484445in}{0.443128in}}{\pgfqpoint{3.728975in}{2.112717in}}%
\pgfusepath{clip}%
\pgfsetbuttcap%
\pgfsetmiterjoin%
\definecolor{currentfill}{rgb}{0.146078,0.146078,0.434314}%
\pgfsetfillcolor{currentfill}%
\pgfsetlinewidth{0.000000pt}%
\definecolor{currentstroke}{rgb}{0.000000,0.000000,0.000000}%
\pgfsetstrokecolor{currentstroke}%
\pgfsetstrokeopacity{0.000000}%
\pgfsetdash{}{0pt}%
\pgfpathmoveto{\pgfqpoint{2.348932in}{0.443128in}}%
\pgfpathlineto{\pgfqpoint{2.846129in}{0.443128in}}%
\pgfpathlineto{\pgfqpoint{2.846129in}{1.809444in}}%
\pgfpathlineto{\pgfqpoint{2.348932in}{1.809444in}}%
\pgfpathclose%
\pgfusepath{fill}%
\end{pgfscope}%
\begin{pgfscope}%
\pgfpathrectangle{\pgfqpoint{0.484445in}{0.443128in}}{\pgfqpoint{3.728975in}{2.112717in}}%
\pgfusepath{clip}%
\pgfsetbuttcap%
\pgfsetmiterjoin%
\definecolor{currentfill}{rgb}{0.146078,0.146078,0.434314}%
\pgfsetfillcolor{currentfill}%
\pgfsetlinewidth{0.000000pt}%
\definecolor{currentstroke}{rgb}{0.000000,0.000000,0.000000}%
\pgfsetstrokecolor{currentstroke}%
\pgfsetstrokeopacity{0.000000}%
\pgfsetdash{}{0pt}%
\pgfpathmoveto{\pgfqpoint{3.591924in}{0.443128in}}%
\pgfpathlineto{\pgfqpoint{4.089121in}{0.443128in}}%
\pgfpathlineto{\pgfqpoint{4.089121in}{1.691909in}}%
\pgfpathlineto{\pgfqpoint{3.591924in}{1.691909in}}%
\pgfpathclose%
\pgfusepath{fill}%
\end{pgfscope}%
\begin{pgfscope}%
\pgfsetbuttcap%
\pgfsetroundjoin%
\definecolor{currentfill}{rgb}{0.000000,0.000000,0.000000}%
\pgfsetfillcolor{currentfill}%
\pgfsetlinewidth{0.803000pt}%
\definecolor{currentstroke}{rgb}{0.000000,0.000000,0.000000}%
\pgfsetstrokecolor{currentstroke}%
\pgfsetdash{}{0pt}%
\pgfsys@defobject{currentmarker}{\pgfqpoint{0.000000in}{-0.048611in}}{\pgfqpoint{0.000000in}{0.000000in}}{%
\pgfpathmoveto{\pgfqpoint{0.000000in}{0.000000in}}%
\pgfpathlineto{\pgfqpoint{0.000000in}{-0.048611in}}%
\pgfusepath{stroke,fill}%
}%
\begin{pgfscope}%
\pgfsys@transformshift{1.105941in}{0.443128in}%
\pgfsys@useobject{currentmarker}{}%
\end{pgfscope}%
\end{pgfscope}%
\begin{pgfscope}%
\definecolor{textcolor}{rgb}{0.000000,0.000000,0.000000}%
\pgfsetstrokecolor{textcolor}%
\pgfsetfillcolor{textcolor}%
\pgftext[x=1.105941in,y=0.345906in,,top]{\color{textcolor}\rmfamily\fontsize{10.000000}{12.000000}\selectfont \(\displaystyle \mathcal{D}_T^{10\%}\)}%
\end{pgfscope}%
\begin{pgfscope}%
\pgfsetbuttcap%
\pgfsetroundjoin%
\definecolor{currentfill}{rgb}{0.000000,0.000000,0.000000}%
\pgfsetfillcolor{currentfill}%
\pgfsetlinewidth{0.803000pt}%
\definecolor{currentstroke}{rgb}{0.000000,0.000000,0.000000}%
\pgfsetstrokecolor{currentstroke}%
\pgfsetdash{}{0pt}%
\pgfsys@defobject{currentmarker}{\pgfqpoint{0.000000in}{-0.048611in}}{\pgfqpoint{0.000000in}{0.000000in}}{%
\pgfpathmoveto{\pgfqpoint{0.000000in}{0.000000in}}%
\pgfpathlineto{\pgfqpoint{0.000000in}{-0.048611in}}%
\pgfusepath{stroke,fill}%
}%
\begin{pgfscope}%
\pgfsys@transformshift{2.348932in}{0.443128in}%
\pgfsys@useobject{currentmarker}{}%
\end{pgfscope}%
\end{pgfscope}%
\begin{pgfscope}%
\definecolor{textcolor}{rgb}{0.000000,0.000000,0.000000}%
\pgfsetstrokecolor{textcolor}%
\pgfsetfillcolor{textcolor}%
\pgftext[x=2.348932in,y=0.345906in,,top]{\color{textcolor}\rmfamily\fontsize{10.000000}{12.000000}\selectfont \(\displaystyle \mathcal{D}_T^\text{Rand.}\)}%
\end{pgfscope}%
\begin{pgfscope}%
\pgfsetbuttcap%
\pgfsetroundjoin%
\definecolor{currentfill}{rgb}{0.000000,0.000000,0.000000}%
\pgfsetfillcolor{currentfill}%
\pgfsetlinewidth{0.803000pt}%
\definecolor{currentstroke}{rgb}{0.000000,0.000000,0.000000}%
\pgfsetstrokecolor{currentstroke}%
\pgfsetdash{}{0pt}%
\pgfsys@defobject{currentmarker}{\pgfqpoint{0.000000in}{-0.048611in}}{\pgfqpoint{0.000000in}{0.000000in}}{%
\pgfpathmoveto{\pgfqpoint{0.000000in}{0.000000in}}%
\pgfpathlineto{\pgfqpoint{0.000000in}{-0.048611in}}%
\pgfusepath{stroke,fill}%
}%
\begin{pgfscope}%
\pgfsys@transformshift{3.591924in}{0.443128in}%
\pgfsys@useobject{currentmarker}{}%
\end{pgfscope}%
\end{pgfscope}%
\begin{pgfscope}%
\definecolor{textcolor}{rgb}{0.000000,0.000000,0.000000}%
\pgfsetstrokecolor{textcolor}%
\pgfsetfillcolor{textcolor}%
\pgftext[x=3.591924in,y=0.345906in,,top]{\color{textcolor}\rmfamily\fontsize{10.000000}{12.000000}\selectfont \(\displaystyle \mathcal{D}_T^\text{Gen.}\)}%
\end{pgfscope}%
\begin{pgfscope}%
\definecolor{textcolor}{rgb}{0.000000,0.000000,0.000000}%
\pgfsetstrokecolor{textcolor}%
\pgfsetfillcolor{textcolor}%
\pgftext[x=2.348933in,y=0.123333in,,top]{\color{textcolor}\rmfamily\fontsize{10.000000}{12.000000}\selectfont Transfer dataset}%
\end{pgfscope}%
\begin{pgfscope}%
\pgfpathrectangle{\pgfqpoint{0.484445in}{0.443128in}}{\pgfqpoint{3.728975in}{2.112717in}}%
\pgfusepath{clip}%
\pgfsetrectcap%
\pgfsetroundjoin%
\pgfsetlinewidth{0.803000pt}%
\definecolor{currentstroke}{rgb}{0.690196,0.690196,0.690196}%
\pgfsetstrokecolor{currentstroke}%
\pgfsetdash{}{0pt}%
\pgfpathmoveto{\pgfqpoint{0.484445in}{0.443128in}}%
\pgfpathlineto{\pgfqpoint{4.213420in}{0.443128in}}%
\pgfusepath{stroke}%
\end{pgfscope}%
\begin{pgfscope}%
\pgfsetbuttcap%
\pgfsetroundjoin%
\definecolor{currentfill}{rgb}{0.000000,0.000000,0.000000}%
\pgfsetfillcolor{currentfill}%
\pgfsetlinewidth{0.803000pt}%
\definecolor{currentstroke}{rgb}{0.000000,0.000000,0.000000}%
\pgfsetstrokecolor{currentstroke}%
\pgfsetdash{}{0pt}%
\pgfsys@defobject{currentmarker}{\pgfqpoint{-0.048611in}{0.000000in}}{\pgfqpoint{-0.000000in}{0.000000in}}{%
\pgfpathmoveto{\pgfqpoint{-0.000000in}{0.000000in}}%
\pgfpathlineto{\pgfqpoint{-0.048611in}{0.000000in}}%
\pgfusepath{stroke,fill}%
}%
\begin{pgfscope}%
\pgfsys@transformshift{0.484445in}{0.443128in}%
\pgfsys@useobject{currentmarker}{}%
\end{pgfscope}%
\end{pgfscope}%
\begin{pgfscope}%
\definecolor{textcolor}{rgb}{0.000000,0.000000,0.000000}%
\pgfsetstrokecolor{textcolor}%
\pgfsetfillcolor{textcolor}%
\pgftext[x=0.317778in, y=0.394934in, left, base]{\color{textcolor}\rmfamily\fontsize{10.000000}{12.000000}\selectfont \(\displaystyle {0}\)}%
\end{pgfscope}%
\begin{pgfscope}%
\pgfpathrectangle{\pgfqpoint{0.484445in}{0.443128in}}{\pgfqpoint{3.728975in}{2.112717in}}%
\pgfusepath{clip}%
\pgfsetrectcap%
\pgfsetroundjoin%
\pgfsetlinewidth{0.803000pt}%
\definecolor{currentstroke}{rgb}{0.690196,0.690196,0.690196}%
\pgfsetstrokecolor{currentstroke}%
\pgfsetdash{}{0pt}%
\pgfpathmoveto{\pgfqpoint{0.484445in}{0.654400in}}%
\pgfpathlineto{\pgfqpoint{4.213420in}{0.654400in}}%
\pgfusepath{stroke}%
\end{pgfscope}%
\begin{pgfscope}%
\pgfsetbuttcap%
\pgfsetroundjoin%
\definecolor{currentfill}{rgb}{0.000000,0.000000,0.000000}%
\pgfsetfillcolor{currentfill}%
\pgfsetlinewidth{0.803000pt}%
\definecolor{currentstroke}{rgb}{0.000000,0.000000,0.000000}%
\pgfsetstrokecolor{currentstroke}%
\pgfsetdash{}{0pt}%
\pgfsys@defobject{currentmarker}{\pgfqpoint{-0.048611in}{0.000000in}}{\pgfqpoint{-0.000000in}{0.000000in}}{%
\pgfpathmoveto{\pgfqpoint{-0.000000in}{0.000000in}}%
\pgfpathlineto{\pgfqpoint{-0.048611in}{0.000000in}}%
\pgfusepath{stroke,fill}%
}%
\begin{pgfscope}%
\pgfsys@transformshift{0.484445in}{0.654400in}%
\pgfsys@useobject{currentmarker}{}%
\end{pgfscope}%
\end{pgfscope}%
\begin{pgfscope}%
\definecolor{textcolor}{rgb}{0.000000,0.000000,0.000000}%
\pgfsetstrokecolor{textcolor}%
\pgfsetfillcolor{textcolor}%
\pgftext[x=0.248333in, y=0.606206in, left, base]{\color{textcolor}\rmfamily\fontsize{10.000000}{12.000000}\selectfont \(\displaystyle {10}\)}%
\end{pgfscope}%
\begin{pgfscope}%
\pgfpathrectangle{\pgfqpoint{0.484445in}{0.443128in}}{\pgfqpoint{3.728975in}{2.112717in}}%
\pgfusepath{clip}%
\pgfsetrectcap%
\pgfsetroundjoin%
\pgfsetlinewidth{0.803000pt}%
\definecolor{currentstroke}{rgb}{0.690196,0.690196,0.690196}%
\pgfsetstrokecolor{currentstroke}%
\pgfsetdash{}{0pt}%
\pgfpathmoveto{\pgfqpoint{0.484445in}{0.865672in}}%
\pgfpathlineto{\pgfqpoint{4.213420in}{0.865672in}}%
\pgfusepath{stroke}%
\end{pgfscope}%
\begin{pgfscope}%
\pgfsetbuttcap%
\pgfsetroundjoin%
\definecolor{currentfill}{rgb}{0.000000,0.000000,0.000000}%
\pgfsetfillcolor{currentfill}%
\pgfsetlinewidth{0.803000pt}%
\definecolor{currentstroke}{rgb}{0.000000,0.000000,0.000000}%
\pgfsetstrokecolor{currentstroke}%
\pgfsetdash{}{0pt}%
\pgfsys@defobject{currentmarker}{\pgfqpoint{-0.048611in}{0.000000in}}{\pgfqpoint{-0.000000in}{0.000000in}}{%
\pgfpathmoveto{\pgfqpoint{-0.000000in}{0.000000in}}%
\pgfpathlineto{\pgfqpoint{-0.048611in}{0.000000in}}%
\pgfusepath{stroke,fill}%
}%
\begin{pgfscope}%
\pgfsys@transformshift{0.484445in}{0.865672in}%
\pgfsys@useobject{currentmarker}{}%
\end{pgfscope}%
\end{pgfscope}%
\begin{pgfscope}%
\definecolor{textcolor}{rgb}{0.000000,0.000000,0.000000}%
\pgfsetstrokecolor{textcolor}%
\pgfsetfillcolor{textcolor}%
\pgftext[x=0.248333in, y=0.817477in, left, base]{\color{textcolor}\rmfamily\fontsize{10.000000}{12.000000}\selectfont \(\displaystyle {20}\)}%
\end{pgfscope}%
\begin{pgfscope}%
\pgfpathrectangle{\pgfqpoint{0.484445in}{0.443128in}}{\pgfqpoint{3.728975in}{2.112717in}}%
\pgfusepath{clip}%
\pgfsetrectcap%
\pgfsetroundjoin%
\pgfsetlinewidth{0.803000pt}%
\definecolor{currentstroke}{rgb}{0.690196,0.690196,0.690196}%
\pgfsetstrokecolor{currentstroke}%
\pgfsetdash{}{0pt}%
\pgfpathmoveto{\pgfqpoint{0.484445in}{1.076944in}}%
\pgfpathlineto{\pgfqpoint{4.213420in}{1.076944in}}%
\pgfusepath{stroke}%
\end{pgfscope}%
\begin{pgfscope}%
\pgfsetbuttcap%
\pgfsetroundjoin%
\definecolor{currentfill}{rgb}{0.000000,0.000000,0.000000}%
\pgfsetfillcolor{currentfill}%
\pgfsetlinewidth{0.803000pt}%
\definecolor{currentstroke}{rgb}{0.000000,0.000000,0.000000}%
\pgfsetstrokecolor{currentstroke}%
\pgfsetdash{}{0pt}%
\pgfsys@defobject{currentmarker}{\pgfqpoint{-0.048611in}{0.000000in}}{\pgfqpoint{-0.000000in}{0.000000in}}{%
\pgfpathmoveto{\pgfqpoint{-0.000000in}{0.000000in}}%
\pgfpathlineto{\pgfqpoint{-0.048611in}{0.000000in}}%
\pgfusepath{stroke,fill}%
}%
\begin{pgfscope}%
\pgfsys@transformshift{0.484445in}{1.076944in}%
\pgfsys@useobject{currentmarker}{}%
\end{pgfscope}%
\end{pgfscope}%
\begin{pgfscope}%
\definecolor{textcolor}{rgb}{0.000000,0.000000,0.000000}%
\pgfsetstrokecolor{textcolor}%
\pgfsetfillcolor{textcolor}%
\pgftext[x=0.248333in, y=1.028749in, left, base]{\color{textcolor}\rmfamily\fontsize{10.000000}{12.000000}\selectfont \(\displaystyle {30}\)}%
\end{pgfscope}%
\begin{pgfscope}%
\pgfpathrectangle{\pgfqpoint{0.484445in}{0.443128in}}{\pgfqpoint{3.728975in}{2.112717in}}%
\pgfusepath{clip}%
\pgfsetrectcap%
\pgfsetroundjoin%
\pgfsetlinewidth{0.803000pt}%
\definecolor{currentstroke}{rgb}{0.690196,0.690196,0.690196}%
\pgfsetstrokecolor{currentstroke}%
\pgfsetdash{}{0pt}%
\pgfpathmoveto{\pgfqpoint{0.484445in}{1.288215in}}%
\pgfpathlineto{\pgfqpoint{4.213420in}{1.288215in}}%
\pgfusepath{stroke}%
\end{pgfscope}%
\begin{pgfscope}%
\pgfsetbuttcap%
\pgfsetroundjoin%
\definecolor{currentfill}{rgb}{0.000000,0.000000,0.000000}%
\pgfsetfillcolor{currentfill}%
\pgfsetlinewidth{0.803000pt}%
\definecolor{currentstroke}{rgb}{0.000000,0.000000,0.000000}%
\pgfsetstrokecolor{currentstroke}%
\pgfsetdash{}{0pt}%
\pgfsys@defobject{currentmarker}{\pgfqpoint{-0.048611in}{0.000000in}}{\pgfqpoint{-0.000000in}{0.000000in}}{%
\pgfpathmoveto{\pgfqpoint{-0.000000in}{0.000000in}}%
\pgfpathlineto{\pgfqpoint{-0.048611in}{0.000000in}}%
\pgfusepath{stroke,fill}%
}%
\begin{pgfscope}%
\pgfsys@transformshift{0.484445in}{1.288215in}%
\pgfsys@useobject{currentmarker}{}%
\end{pgfscope}%
\end{pgfscope}%
\begin{pgfscope}%
\definecolor{textcolor}{rgb}{0.000000,0.000000,0.000000}%
\pgfsetstrokecolor{textcolor}%
\pgfsetfillcolor{textcolor}%
\pgftext[x=0.248333in, y=1.240021in, left, base]{\color{textcolor}\rmfamily\fontsize{10.000000}{12.000000}\selectfont \(\displaystyle {40}\)}%
\end{pgfscope}%
\begin{pgfscope}%
\pgfpathrectangle{\pgfqpoint{0.484445in}{0.443128in}}{\pgfqpoint{3.728975in}{2.112717in}}%
\pgfusepath{clip}%
\pgfsetrectcap%
\pgfsetroundjoin%
\pgfsetlinewidth{0.803000pt}%
\definecolor{currentstroke}{rgb}{0.690196,0.690196,0.690196}%
\pgfsetstrokecolor{currentstroke}%
\pgfsetdash{}{0pt}%
\pgfpathmoveto{\pgfqpoint{0.484445in}{1.499487in}}%
\pgfpathlineto{\pgfqpoint{4.213420in}{1.499487in}}%
\pgfusepath{stroke}%
\end{pgfscope}%
\begin{pgfscope}%
\pgfsetbuttcap%
\pgfsetroundjoin%
\definecolor{currentfill}{rgb}{0.000000,0.000000,0.000000}%
\pgfsetfillcolor{currentfill}%
\pgfsetlinewidth{0.803000pt}%
\definecolor{currentstroke}{rgb}{0.000000,0.000000,0.000000}%
\pgfsetstrokecolor{currentstroke}%
\pgfsetdash{}{0pt}%
\pgfsys@defobject{currentmarker}{\pgfqpoint{-0.048611in}{0.000000in}}{\pgfqpoint{-0.000000in}{0.000000in}}{%
\pgfpathmoveto{\pgfqpoint{-0.000000in}{0.000000in}}%
\pgfpathlineto{\pgfqpoint{-0.048611in}{0.000000in}}%
\pgfusepath{stroke,fill}%
}%
\begin{pgfscope}%
\pgfsys@transformshift{0.484445in}{1.499487in}%
\pgfsys@useobject{currentmarker}{}%
\end{pgfscope}%
\end{pgfscope}%
\begin{pgfscope}%
\definecolor{textcolor}{rgb}{0.000000,0.000000,0.000000}%
\pgfsetstrokecolor{textcolor}%
\pgfsetfillcolor{textcolor}%
\pgftext[x=0.248333in, y=1.451293in, left, base]{\color{textcolor}\rmfamily\fontsize{10.000000}{12.000000}\selectfont \(\displaystyle {50}\)}%
\end{pgfscope}%
\begin{pgfscope}%
\pgfpathrectangle{\pgfqpoint{0.484445in}{0.443128in}}{\pgfqpoint{3.728975in}{2.112717in}}%
\pgfusepath{clip}%
\pgfsetrectcap%
\pgfsetroundjoin%
\pgfsetlinewidth{0.803000pt}%
\definecolor{currentstroke}{rgb}{0.690196,0.690196,0.690196}%
\pgfsetstrokecolor{currentstroke}%
\pgfsetdash{}{0pt}%
\pgfpathmoveto{\pgfqpoint{0.484445in}{1.710759in}}%
\pgfpathlineto{\pgfqpoint{4.213420in}{1.710759in}}%
\pgfusepath{stroke}%
\end{pgfscope}%
\begin{pgfscope}%
\pgfsetbuttcap%
\pgfsetroundjoin%
\definecolor{currentfill}{rgb}{0.000000,0.000000,0.000000}%
\pgfsetfillcolor{currentfill}%
\pgfsetlinewidth{0.803000pt}%
\definecolor{currentstroke}{rgb}{0.000000,0.000000,0.000000}%
\pgfsetstrokecolor{currentstroke}%
\pgfsetdash{}{0pt}%
\pgfsys@defobject{currentmarker}{\pgfqpoint{-0.048611in}{0.000000in}}{\pgfqpoint{-0.000000in}{0.000000in}}{%
\pgfpathmoveto{\pgfqpoint{-0.000000in}{0.000000in}}%
\pgfpathlineto{\pgfqpoint{-0.048611in}{0.000000in}}%
\pgfusepath{stroke,fill}%
}%
\begin{pgfscope}%
\pgfsys@transformshift{0.484445in}{1.710759in}%
\pgfsys@useobject{currentmarker}{}%
\end{pgfscope}%
\end{pgfscope}%
\begin{pgfscope}%
\definecolor{textcolor}{rgb}{0.000000,0.000000,0.000000}%
\pgfsetstrokecolor{textcolor}%
\pgfsetfillcolor{textcolor}%
\pgftext[x=0.248333in, y=1.662564in, left, base]{\color{textcolor}\rmfamily\fontsize{10.000000}{12.000000}\selectfont \(\displaystyle {60}\)}%
\end{pgfscope}%
\begin{pgfscope}%
\pgfpathrectangle{\pgfqpoint{0.484445in}{0.443128in}}{\pgfqpoint{3.728975in}{2.112717in}}%
\pgfusepath{clip}%
\pgfsetrectcap%
\pgfsetroundjoin%
\pgfsetlinewidth{0.803000pt}%
\definecolor{currentstroke}{rgb}{0.690196,0.690196,0.690196}%
\pgfsetstrokecolor{currentstroke}%
\pgfsetdash{}{0pt}%
\pgfpathmoveto{\pgfqpoint{0.484445in}{1.922030in}}%
\pgfpathlineto{\pgfqpoint{4.213420in}{1.922030in}}%
\pgfusepath{stroke}%
\end{pgfscope}%
\begin{pgfscope}%
\pgfsetbuttcap%
\pgfsetroundjoin%
\definecolor{currentfill}{rgb}{0.000000,0.000000,0.000000}%
\pgfsetfillcolor{currentfill}%
\pgfsetlinewidth{0.803000pt}%
\definecolor{currentstroke}{rgb}{0.000000,0.000000,0.000000}%
\pgfsetstrokecolor{currentstroke}%
\pgfsetdash{}{0pt}%
\pgfsys@defobject{currentmarker}{\pgfqpoint{-0.048611in}{0.000000in}}{\pgfqpoint{-0.000000in}{0.000000in}}{%
\pgfpathmoveto{\pgfqpoint{-0.000000in}{0.000000in}}%
\pgfpathlineto{\pgfqpoint{-0.048611in}{0.000000in}}%
\pgfusepath{stroke,fill}%
}%
\begin{pgfscope}%
\pgfsys@transformshift{0.484445in}{1.922030in}%
\pgfsys@useobject{currentmarker}{}%
\end{pgfscope}%
\end{pgfscope}%
\begin{pgfscope}%
\definecolor{textcolor}{rgb}{0.000000,0.000000,0.000000}%
\pgfsetstrokecolor{textcolor}%
\pgfsetfillcolor{textcolor}%
\pgftext[x=0.248333in, y=1.873836in, left, base]{\color{textcolor}\rmfamily\fontsize{10.000000}{12.000000}\selectfont \(\displaystyle {70}\)}%
\end{pgfscope}%
\begin{pgfscope}%
\pgfpathrectangle{\pgfqpoint{0.484445in}{0.443128in}}{\pgfqpoint{3.728975in}{2.112717in}}%
\pgfusepath{clip}%
\pgfsetrectcap%
\pgfsetroundjoin%
\pgfsetlinewidth{0.803000pt}%
\definecolor{currentstroke}{rgb}{0.690196,0.690196,0.690196}%
\pgfsetstrokecolor{currentstroke}%
\pgfsetdash{}{0pt}%
\pgfpathmoveto{\pgfqpoint{0.484445in}{2.133302in}}%
\pgfpathlineto{\pgfqpoint{4.213420in}{2.133302in}}%
\pgfusepath{stroke}%
\end{pgfscope}%
\begin{pgfscope}%
\pgfsetbuttcap%
\pgfsetroundjoin%
\definecolor{currentfill}{rgb}{0.000000,0.000000,0.000000}%
\pgfsetfillcolor{currentfill}%
\pgfsetlinewidth{0.803000pt}%
\definecolor{currentstroke}{rgb}{0.000000,0.000000,0.000000}%
\pgfsetstrokecolor{currentstroke}%
\pgfsetdash{}{0pt}%
\pgfsys@defobject{currentmarker}{\pgfqpoint{-0.048611in}{0.000000in}}{\pgfqpoint{-0.000000in}{0.000000in}}{%
\pgfpathmoveto{\pgfqpoint{-0.000000in}{0.000000in}}%
\pgfpathlineto{\pgfqpoint{-0.048611in}{0.000000in}}%
\pgfusepath{stroke,fill}%
}%
\begin{pgfscope}%
\pgfsys@transformshift{0.484445in}{2.133302in}%
\pgfsys@useobject{currentmarker}{}%
\end{pgfscope}%
\end{pgfscope}%
\begin{pgfscope}%
\definecolor{textcolor}{rgb}{0.000000,0.000000,0.000000}%
\pgfsetstrokecolor{textcolor}%
\pgfsetfillcolor{textcolor}%
\pgftext[x=0.248333in, y=2.085108in, left, base]{\color{textcolor}\rmfamily\fontsize{10.000000}{12.000000}\selectfont \(\displaystyle {80}\)}%
\end{pgfscope}%
\begin{pgfscope}%
\pgfpathrectangle{\pgfqpoint{0.484445in}{0.443128in}}{\pgfqpoint{3.728975in}{2.112717in}}%
\pgfusepath{clip}%
\pgfsetrectcap%
\pgfsetroundjoin%
\pgfsetlinewidth{0.803000pt}%
\definecolor{currentstroke}{rgb}{0.690196,0.690196,0.690196}%
\pgfsetstrokecolor{currentstroke}%
\pgfsetdash{}{0pt}%
\pgfpathmoveto{\pgfqpoint{0.484445in}{2.344574in}}%
\pgfpathlineto{\pgfqpoint{4.213420in}{2.344574in}}%
\pgfusepath{stroke}%
\end{pgfscope}%
\begin{pgfscope}%
\pgfsetbuttcap%
\pgfsetroundjoin%
\definecolor{currentfill}{rgb}{0.000000,0.000000,0.000000}%
\pgfsetfillcolor{currentfill}%
\pgfsetlinewidth{0.803000pt}%
\definecolor{currentstroke}{rgb}{0.000000,0.000000,0.000000}%
\pgfsetstrokecolor{currentstroke}%
\pgfsetdash{}{0pt}%
\pgfsys@defobject{currentmarker}{\pgfqpoint{-0.048611in}{0.000000in}}{\pgfqpoint{-0.000000in}{0.000000in}}{%
\pgfpathmoveto{\pgfqpoint{-0.000000in}{0.000000in}}%
\pgfpathlineto{\pgfqpoint{-0.048611in}{0.000000in}}%
\pgfusepath{stroke,fill}%
}%
\begin{pgfscope}%
\pgfsys@transformshift{0.484445in}{2.344574in}%
\pgfsys@useobject{currentmarker}{}%
\end{pgfscope}%
\end{pgfscope}%
\begin{pgfscope}%
\definecolor{textcolor}{rgb}{0.000000,0.000000,0.000000}%
\pgfsetstrokecolor{textcolor}%
\pgfsetfillcolor{textcolor}%
\pgftext[x=0.248333in, y=2.296379in, left, base]{\color{textcolor}\rmfamily\fontsize{10.000000}{12.000000}\selectfont \(\displaystyle {90}\)}%
\end{pgfscope}%
\begin{pgfscope}%
\pgfpathrectangle{\pgfqpoint{0.484445in}{0.443128in}}{\pgfqpoint{3.728975in}{2.112717in}}%
\pgfusepath{clip}%
\pgfsetrectcap%
\pgfsetroundjoin%
\pgfsetlinewidth{0.803000pt}%
\definecolor{currentstroke}{rgb}{0.690196,0.690196,0.690196}%
\pgfsetstrokecolor{currentstroke}%
\pgfsetdash{}{0pt}%
\pgfpathmoveto{\pgfqpoint{0.484445in}{2.555846in}}%
\pgfpathlineto{\pgfqpoint{4.213420in}{2.555846in}}%
\pgfusepath{stroke}%
\end{pgfscope}%
\begin{pgfscope}%
\pgfsetbuttcap%
\pgfsetroundjoin%
\definecolor{currentfill}{rgb}{0.000000,0.000000,0.000000}%
\pgfsetfillcolor{currentfill}%
\pgfsetlinewidth{0.803000pt}%
\definecolor{currentstroke}{rgb}{0.000000,0.000000,0.000000}%
\pgfsetstrokecolor{currentstroke}%
\pgfsetdash{}{0pt}%
\pgfsys@defobject{currentmarker}{\pgfqpoint{-0.048611in}{0.000000in}}{\pgfqpoint{-0.000000in}{0.000000in}}{%
\pgfpathmoveto{\pgfqpoint{-0.000000in}{0.000000in}}%
\pgfpathlineto{\pgfqpoint{-0.048611in}{0.000000in}}%
\pgfusepath{stroke,fill}%
}%
\begin{pgfscope}%
\pgfsys@transformshift{0.484445in}{2.555846in}%
\pgfsys@useobject{currentmarker}{}%
\end{pgfscope}%
\end{pgfscope}%
\begin{pgfscope}%
\definecolor{textcolor}{rgb}{0.000000,0.000000,0.000000}%
\pgfsetstrokecolor{textcolor}%
\pgfsetfillcolor{textcolor}%
\pgftext[x=0.178889in, y=2.507651in, left, base]{\color{textcolor}\rmfamily\fontsize{10.000000}{12.000000}\selectfont \(\displaystyle {100}\)}%
\end{pgfscope}%
\begin{pgfscope}%
\definecolor{textcolor}{rgb}{0.000000,0.000000,0.000000}%
\pgfsetstrokecolor{textcolor}%
\pgfsetfillcolor{textcolor}%
\pgftext[x=0.123333in,y=1.499487in,,bottom,rotate=90.000000]{\color{textcolor}\rmfamily\fontsize{10.000000}{12.000000}\selectfont Score}%
\end{pgfscope}%
\begin{pgfscope}%
\pgfsetrectcap%
\pgfsetmiterjoin%
\pgfsetlinewidth{0.803000pt}%
\definecolor{currentstroke}{rgb}{0.000000,0.000000,0.000000}%
\pgfsetstrokecolor{currentstroke}%
\pgfsetdash{}{0pt}%
\pgfpathmoveto{\pgfqpoint{0.484445in}{0.443128in}}%
\pgfpathlineto{\pgfqpoint{0.484445in}{2.555846in}}%
\pgfusepath{stroke}%
\end{pgfscope}%
\begin{pgfscope}%
\pgfsetrectcap%
\pgfsetmiterjoin%
\pgfsetlinewidth{0.803000pt}%
\definecolor{currentstroke}{rgb}{0.000000,0.000000,0.000000}%
\pgfsetstrokecolor{currentstroke}%
\pgfsetdash{}{0pt}%
\pgfpathmoveto{\pgfqpoint{4.213420in}{0.443128in}}%
\pgfpathlineto{\pgfqpoint{4.213420in}{2.555846in}}%
\pgfusepath{stroke}%
\end{pgfscope}%
\begin{pgfscope}%
\pgfsetrectcap%
\pgfsetmiterjoin%
\pgfsetlinewidth{0.803000pt}%
\definecolor{currentstroke}{rgb}{0.000000,0.000000,0.000000}%
\pgfsetstrokecolor{currentstroke}%
\pgfsetdash{}{0pt}%
\pgfpathmoveto{\pgfqpoint{0.484445in}{0.443128in}}%
\pgfpathlineto{\pgfqpoint{4.213420in}{0.443128in}}%
\pgfusepath{stroke}%
\end{pgfscope}%
\begin{pgfscope}%
\pgfsetrectcap%
\pgfsetmiterjoin%
\pgfsetlinewidth{0.803000pt}%
\definecolor{currentstroke}{rgb}{0.000000,0.000000,0.000000}%
\pgfsetstrokecolor{currentstroke}%
\pgfsetdash{}{0pt}%
\pgfpathmoveto{\pgfqpoint{0.484445in}{2.555846in}}%
\pgfpathlineto{\pgfqpoint{4.213420in}{2.555846in}}%
\pgfusepath{stroke}%
\end{pgfscope}%
\begin{pgfscope}%
\pgfsetbuttcap%
\pgfsetmiterjoin%
\definecolor{currentfill}{rgb}{1.000000,1.000000,1.000000}%
\pgfsetfillcolor{currentfill}%
\pgfsetfillopacity{0.800000}%
\pgfsetlinewidth{1.003750pt}%
\definecolor{currentstroke}{rgb}{0.800000,0.800000,0.800000}%
\pgfsetstrokecolor{currentstroke}%
\pgfsetstrokeopacity{0.800000}%
\pgfsetdash{}{0pt}%
\pgfpathmoveto{\pgfqpoint{3.198975in}{2.055290in}}%
\pgfpathlineto{\pgfqpoint{4.116198in}{2.055290in}}%
\pgfpathquadraticcurveto{\pgfqpoint{4.143976in}{2.055290in}}{\pgfqpoint{4.143976in}{2.083068in}}%
\pgfpathlineto{\pgfqpoint{4.143976in}{2.458623in}}%
\pgfpathquadraticcurveto{\pgfqpoint{4.143976in}{2.486401in}}{\pgfqpoint{4.116198in}{2.486401in}}%
\pgfpathlineto{\pgfqpoint{3.198975in}{2.486401in}}%
\pgfpathquadraticcurveto{\pgfqpoint{3.171198in}{2.486401in}}{\pgfqpoint{3.171198in}{2.458623in}}%
\pgfpathlineto{\pgfqpoint{3.171198in}{2.083068in}}%
\pgfpathquadraticcurveto{\pgfqpoint{3.171198in}{2.055290in}}{\pgfqpoint{3.198975in}{2.055290in}}%
\pgfpathclose%
\pgfusepath{stroke,fill}%
\end{pgfscope}%
\begin{pgfscope}%
\pgfsetbuttcap%
\pgfsetmiterjoin%
\definecolor{currentfill}{rgb}{0.434314,0.146078,0.146078}%
\pgfsetfillcolor{currentfill}%
\pgfsetlinewidth{0.000000pt}%
\definecolor{currentstroke}{rgb}{0.000000,0.000000,0.000000}%
\pgfsetstrokecolor{currentstroke}%
\pgfsetstrokeopacity{0.000000}%
\pgfsetdash{}{0pt}%
\pgfpathmoveto{\pgfqpoint{3.226753in}{2.331401in}}%
\pgfpathlineto{\pgfqpoint{3.504531in}{2.331401in}}%
\pgfpathlineto{\pgfqpoint{3.504531in}{2.428623in}}%
\pgfpathlineto{\pgfqpoint{3.226753in}{2.428623in}}%
\pgfpathclose%
\pgfusepath{fill}%
\end{pgfscope}%
\begin{pgfscope}%
\definecolor{textcolor}{rgb}{0.000000,0.000000,0.000000}%
\pgfsetstrokecolor{textcolor}%
\pgfsetfillcolor{textcolor}%
\pgftext[x=3.615642in,y=2.331401in,left,base]{\color{textcolor}\rmfamily\fontsize{10.000000}{12.000000}\selectfont SQuAD}%
\end{pgfscope}%
\begin{pgfscope}%
\pgfsetbuttcap%
\pgfsetmiterjoin%
\definecolor{currentfill}{rgb}{0.146078,0.146078,0.434314}%
\pgfsetfillcolor{currentfill}%
\pgfsetlinewidth{0.000000pt}%
\definecolor{currentstroke}{rgb}{0.000000,0.000000,0.000000}%
\pgfsetstrokecolor{currentstroke}%
\pgfsetstrokeopacity{0.000000}%
\pgfsetdash{}{0pt}%
\pgfpathmoveto{\pgfqpoint{3.226753in}{2.137790in}}%
\pgfpathlineto{\pgfqpoint{3.504531in}{2.137790in}}%
\pgfpathlineto{\pgfqpoint{3.504531in}{2.235012in}}%
\pgfpathlineto{\pgfqpoint{3.226753in}{2.235012in}}%
\pgfpathclose%
\pgfusepath{fill}%
\end{pgfscope}%
\begin{pgfscope}%
\definecolor{textcolor}{rgb}{0.000000,0.000000,0.000000}%
\pgfsetstrokecolor{textcolor}%
\pgfsetfillcolor{textcolor}%
\pgftext[x=3.615642in,y=2.137790in,left,base]{\color{textcolor}\rmfamily\fontsize{10.000000}{12.000000}\selectfont GLUE}%
\end{pgfscope}%
\end{pgfpicture}%
\makeatother%
\endgroup%

            \end{center}
            \caption{Plot of performance of \bertstudent on GLUE and SQuAD using transfer datasets of various compositions.}
        \end{figure}
    \end{frame}
    
    \section{Conclusion}
    \begin{frame}{Summary of results}
        \begin{itemize}
            \item When it comes to the \emph{size} of the transfer dataset $\dataset_T$, we observed competitive performance on GLUE and acceptable performance on SQuAD when a transfer dataset is used of ``only'' $\sim \SI{100}{\kilo\nothing}$ sequences or $\sim \SI{4}{\mega\nothing}$ tokens. Whereas performance on GLUE does not seem to benefit much from using a larger transfer dataset, performance on SQuAD does. 
            \item With respect to the \emph{composition} of the transfer dataset $\dataset_T$, we observed diminishing (though still acceptable) performance on GLUE and poor performance on SQuAD when a transfer dataset is used that does not consist of proper \gls{nl} (i.e. randomized and/or generated) data. More surprisingly, we observe diminishing performance on GLUE when a less ``informative'' transfer dataset is used for \gls{kd}, whereas for SQuAD, we observe an increase in performance.
        \end{itemize}
    \end{frame}
    
    \begin{frame}{Answer to Research Question 1}
        \begin{itemize}
            \item The general effect on downstream task performance of using a smaller transfer dataset $\dataset_T$ to distill the knowledge of \bertbase into \bertstudent is a \emph{slight decrease} in performance.
            \item We consider the decrease in performance on GLUE to be marginal, especially if you also weigh the costs (both financially and environmentally, as well as temporally).
            \item Performance on SQuAD benefits significantly more from using a (relatively) larger transfer dataset $\dataset_T$ during pre-training in the \gls{kd} process.
        \end{itemize}
    \end{frame}
    
    \begin{frame}{Answer to Research Question 2}
        \begin{itemize}
            \item The general effect on downstream task performance of using a transfer dataset $\dataset_T$ composed of non \gls{nl} data to distill the knowledge of \bertbase into \bertstudent is a \emph{significant decrease} in performance.
            \item This is the case for GLUE (though performance is still acceptable), but even more so for SQuAD. This result makes intuitive sense, as you would expect that (pre-)training with less informative data would yield a less informed \gls{nn}.
            \item Nonetheless, this answer is by no means conclusive, as there are still aspects unexplored, which we leave for future work. Moreover, for low-resource languages, randomizing and/or generating sequences might still prove worthwhile.
        \end{itemize}
    \end{frame}


    % references
    \renewcommand{\bibsection}{}
    \renewcommand*{\bibfont}{\tiny}
    \DeclareRobustCommand{\VAN}[3]{#3}
    \begin{frame}[allowframebreaks]{References}
        \bibliography{references}
        \bibliographystyle{apalike}
    \end{frame}



    % end frame
    \metroset{numbering=none}
    \begin{frame}[standout]
    \end{frame}
\end{document}